%! Author = linus.lundqvist2
%! Date = 2023-03-17

% Preamble
\documentclass[11pt]{article}

% Packages
\usepackage{amsmath}
\usepackage{graphicx}
\usepackage{biblatex}

% Document
\begin{document}

    \begin{titlepage}
        \begin{center}
            \vspace*{1cm}

            \Huge
            \textbf{Title}

            \vspace{0.5cm}
            \LARGE
            Subtitle

            \vspace{1.5cm}

            \textbf{Author Name}

            \vfill

            Ett PM om energiförsörjning \\
            Fysik 1

            \vspace{0.8cm}

             \includegraphics[width=0.4\textwidth]{C:\Webb\PM-Energi\src\NTI Gymnasiet_Symbol_print_svart.png}

            \Large
            Teknikprogrammet\\
            NTI Gymnasiet\\
            Umeå\\
            \today

        \end{center}
    \end{titlepage}

    \tableofcontents
    \newpage

    \section{Inledning}
    Kärnkraft är den process då man utvinner energi ur långa prosedurer där man splitrar störe ämnen för att utvinna energin inom deras bindningar.

    \subsection{frågeställningar}
    \begin{enumerate}
        \item Hur fungerar ett kärnkraftverk?\\
        Kärnkraft kommer i två varianter, fission och fusion, fission bygger på att man skjuter protoner på störe atomer, vanligast Uranium och Plutonium, vilket påbörjar kedjereaktion där denna process fortsätts av sig själv.\\
        \\
        Fussion på andra sidan är teoretisk talat en potensiell möjlighet i framtiden, i denna process så slår man ihop atomer av mindre storlek för att bilda störe ämnen, genom denna process frigörs energi som går gynna oss av like fission. Detta sagt dock så är denna teknik fortfarande endast inom teorins värld när det kommer till måls att utvinna energi ur processen.
        \item Vilka miljöpåverkan har ett kärnkraftverk lokalt och globalt?\\

        \item Hur påverkar kärnkraft samhället (Ekonomi/politik/konflikter/m.m.) lokalt och globalt?\\

    \end{enumerate}

    \section{Slutsatser}
    Här kan du dra slutsatser eller sammanfatta ditt resultat

    \section{referenser}
    \printbibliography

\end{document}